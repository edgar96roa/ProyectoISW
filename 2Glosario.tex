%==============={Glosario}==========================================
% En esta parte se pretende introducir los conceptos del negocio
%---------------------------------------------------------

\chapter{Glosario}

\begin{description}[style=nextline]

\section{Definciones}

\item[Anuncio Publicitario]
Espacio físico ubicado en una zona estratégica utilizado para fines publicitarios en el cuál un \textbf{espectacular} estará montado.

Existen tres tipos de espacios para anuncios publicitarios, los cuales varían de precio:
\begin{itemize}
    \item \textbf{Vía primaria}: principal tipo de espacio publicitario. Estratégicamente, se encuentra ubicado una zona altamente transitada. Se considera como los que tienen mas impacto publicitario en la población local.
    \item \textbf{Vía secundaria}: Este tipo de anuncio publicitario es transitado y frecuentado por la población local menos frecuente que el de la vía primaria y con menor impacto publicitario.
    \item \textbf{VIP}: Espacios publicitarios exclusivos para cierto subconjunto de la población. Aunque menos frecuentado, puede causar mucho mas impacto para grupos específicos de la población.
\end{itemize}

Puede ser propio de la empresa o rentado por parte de un \textbf{arrendador}.

\item[Espectacular]
Es la muestra publicitaria como tal que se montará en un espacio publicitario. 

\item[Cliente]
Persona física o moral que tiene necesidad de publicar anuncios publicitarios y contrata a la empresa para realizar un contrato.

\item[Contrato]
Acuerdo mutuo entre la empresa dueña de los espectaculares y el cliente que tiene deseos de utilizar el \textbf{anuncio publicitario} para montar un \textbf{espectacular} durante cierto tiempo. El acuerdo será respetado desde el inicio de la fecha del contrato hasta su vencimiento.

\item[Permiso legal]
Permiso especial expedido por medio de un oficio por parte del gobierno en donde se autoriza a la empresa el uso de uno o mas \textbf{anuncios publicitarios} para fines publicitarios. Un \textbf{permiso legal} tiene una fecha de vencimiento, por lo cual el permiso estará válido desde su fecha de expedición hasta su vencimiento. El oficio expedido contiene un \textbf{número de folio.} El incumplimiento de las leyes puede provocar clausuras por parte del gobierno. 

\item[Seguro]
Es un contrato que la compañía Espectaculares S.A de C.V utiliza para realizar la cobertura en caso de algún siniestro en los \textbf{espectaculares}.

\item[Departamento de Ventas]
Es el departamento encargado de planear, ejecutar y controlar las actividades en este campo. Deben de dar seguimiento y control continuo a las actividades de venta.
Tiene como encargado al \textbf{gerente de ventas}. Son los encargados de cerrar un \textbf{contrato} con el \textbf{cliente}.

\item[Gerente de Ventas]
Es la persona a cargo de supervisar y dirigir las actividades del \textbf{departamento de ventas}. Coordina y monitorea el trabajo de los empleados a su cargo.

\item[Departamento de Infraestructura]
El departamento de Infraestructura es el encargado de hacer el manejo de la información respecto a los \textbf{espectaculares}. Así como programar, organizar, controlar, ejecutar y supervisar los diferentes servicios generales de mantenimiento. 

\item[Gerente de Infraestructura]
Es la persona a cargo de supervisar y dirigir las actividades del \textbf{departamento de infraestructura}. Coordina y monitorea el trabajo de los empleados a su cargo.

\item[Departamento de Instalación y Mantenimiento]
Es el departamento encargado de hacer servicios de instalación, mantenimiento preventivo y mantenimiento correctivo a los \textbf{espectaculares}.

\item[Departamento Jurídico]
Son el departamento encargado de pedir la renovación o los \textbf{permisos legales} pertinentes a las delegaciones en las cuales se encuentra cada uno de los espectaculares. 
\item[Arrendador]
Es la persona a la cuál la empresa le renta un espacio físico para instalar un \textbf{espectacular}.

\item[Capital Humano]
Es el departamento que tiene entre sus funciones manejar la información de los \textbf{empleados} dentro de la empresa.

\item[Empleado]
Son las personas que laboran dentro de las instalaciones de la empresa Espectaculares S.A de C.V y que cuentan con un número de empleado.

\item[Penalización]
Es un cobro extra para la empresa o el \textbf{cliente} en caso de incumplir alguna de las clausulas del contrato realizado para la renta de \textbf{espectaculares}.

\end{description}

\newpage

\section{Responsabilidades}

A continuación, se describen la importancia de los usuarios definidos en el glosario anterior, especificando sus responsabilidades entre ellos mismos y con el sistema al mimso tiempo que definimos su participación en distintos procesos.

\begin{Usuario}{\subsection{Cliente}}{
   Definimos al cliente como la persona o empresa externa a la compañía la cual acude a la empresa en búsqueda de una solución de marketing para sus necesidades.
}
    \item[Responsabilidades:]\cdtEmpty 
   \begin{itemize}
       \item Solicitar un plan de marketing para la empresa a la cuál está asociado.
       \item Identificar la cantidad de personas a las cuales el espectacular deberá impactar.
       \item Encargado de definir el presupuesto con el que cuenta para que el vendedor pueda brindarle un plan a su medida.
       \item Respetar al presupuesto acordado con el vendedor.
	   \item Respetar las fechas de inicio y finalización de contrato.
   \end{itemize}
	\item[Procesos en los que participa:] \cdtEmpty
    \begin{itemize}
		\item Contratación, renovación y cancelación de contratos de servicios de marketing.
	    \item Monitoreo de historial y estatus de servicios contratados.
    \end{itemize}
\end{Usuario}

\begin{Usuario}{\subsection{Personal de Ventas}}{
	El personal de ventas es el encargado de mostrar al cliente las diferentes opciones de publicidad que la empresa ofrece, con el fin de hacer que el cliente encuentre alguna que se ajuste a sus necesidades y presupuestos.\\
}
    \item[Responsabilidades:] \cdtEmpty
    \begin{itemize}
		\item Mostrar al cliente las opciones de publicidad con las que la empresa cuenta.
        \item Encargado de la negociación con el cliente.
        \item Cerrar un trato que beneficie al cliente y genere ganancias a la empresa.
        \item Interacción con los posibles clientes para la realización de un contrato.
    \end{itemize}

	\item[Procesos en los que participa:] \cdtEmpty
    \begin{itemize}
		\item Altas, bajas y cambios de información de clientes.
		\item Cotización de contratos.
    \end{itemize}
\end{Usuario}

\begin{Usuario}{\subsection{Gerente de Infraestructura}}{
	El gerente de infraestructura es el encargado de realizar nuevas adquisiciones de espectaculares para la empresa además de actualizar su información pertinente.\\
}
    \item[Responsabilidades:] \cdtEmpty
    \begin{itemize}
        \item Gestionar la información congruente a los espectaculares.
		\item Adquisición de espacios disponibles para la instalación de nuevos espectaculares.
		\item Asignación de tareas al personal de mantenimiento e instalación.
		\item Monitoreo del estatus de los seguros de los espectaculares.
		\item Comunicación continua con las personas arrendadoras de espacios para espectaculares.
	\end{itemize}

	\item[Procesos en los que participa:] \cdtEmpty
    \begin{itemize}
		\item Adquisición de espectaculares.
		\item Elaboración de reporte de incidencias.
		\item Altas, bajas y cambios de información de espectaculares.
		\item Validación de los requerimientos de seguridad con base a los reglamentos pertinentes.
		\item Asignación de personal de instalación y mantenimiento para instalación, desinstalación y mantenimiento preventivo o correctivo.
    \end{itemize}
\end{Usuario}

%\begin{Usuario}{\subsection{Gerente de Ventas}}{
%	Es el encargado de todas las operaciones de ventas al mayoreo y al menudeo. coordina y supervisa el trabajo %de los Agentes de Ventas y Encargados de Tienda.
%	Reporta directamente al Gerente de Operaciones
%}
 %   \item[Responsabilidades:] \cdtEmpty
 %   \begin{itemize}
%		\item Supervisar la operación de ventas.
%		\item Plantear y supervisar el logro de las metas de ventas de la empresa y su crecimiento económico.
%		\item ...
%    \end{itemize}

%	\item[Perfil:] \cdtEmpty
%    \begin{itemize}
%		\item Amplia experiencia en el ramo.
%		\item Licenciatura como mínimo.
%		\item ...
%    \end{itemize}
%	\item[Procesos en los que participa:] \cdtEmpty
 %   \begin{itemize}
%		\item PC-V01 Aprobar las ordenes de compra al mayoreo.
%		\item PC-V02 Supervisar las ventas al menudeo.
%		\item PC-V03 Elaborar informe de ventas mensual.
%		\item ...
%    \end{itemize}
%\end{Usuario}

%---------------------------------------------------------
\begin{Usuario}{\subsection{Personal de Instalación y Mantenimiento}}{
    Es el personal encargado de la mano de obra referente a los espectaculares, es decir instalación, desinstalación, atención a incidentes.\\
}
    \item[Responsabilidades:] \cdtEmpty
    \begin{itemize}
        \item Realización de instalación, desinstalación y mantenimiento a espectaculares.
		\item Notificar los servicios que realiza.
		\item Atención pronta a reportes de incidencias.
    \end{itemize}

	\item[Procesos en los que participa:] \cdtEmpty
    \begin{itemize}
		\item Instalación, desinstalación y mantenimiento a espectaculares.
		\item Actualización del estatus de un espectacular.
    \end{itemize}
\end{Usuario}

\begin{Usuario}{\subsection{Arrendador}}{
    Persona encargada de negociar el uso de un espacio de su propiedad para uso de un espectacular.\\
}
    \item[Responsabilidades:] \cdtEmpty
    \begin{itemize}
        \item Revisión constante del estado de los espectaculares que tenga a su nombre en un contrato de arrendamiento.
        \item Reportar algún incidente relacionado con el espectacular.
        \item Respetar los plazos establecidos en la renta del espacio para el espectacular.
    \end{itemize}
    
    \item[Procesos en los que participa:] \cdtEmpty
    \begin{itemize}
        \item Reporte de desperfectos en espectaculares.
        %\item Validación del estatus de un espectacular.
    \end{itemize}
\end{Usuario}

\begin{Usuario}{\subsection{Capital humano}}{
La responsabilidad de capital humano sera la gestión de información de empleados que requieren tener acceso al sistema.
}

\item[Responsabilidades:]\cdtEmpty
\begin{itemize}
    \item Manejo de información congruente al personal de la empresa.
\end{itemize}
\item[Procesos en los que participa:]\cdtEmpty
\begin{itemize}
    \item Responsable de creación de usuarios al personal dentro de la empresa que necesite acceder al sistema.
    \item Responsable de cambios y eliminación de información correspondiente de los empleados dentro del sistema.
\end{itemize}
\end{Usuario}

\begin{Usuario}{\subsection{Departamento Jurídico}}{
La actividad que desempeña el área jurídica esta relacionada con los permisos y seguros relacionados a los espectaculares.
}
\item[Responsabilidades:]\cdtEmpty
\begin{itemize}
    \item Gestión de los permisos y seguros de espectaculares.
    \item Garantizar el cumplimiento de las regulaciones y las leyes.
    \item Representar a la empresa en asuntos legales de diversa índole ante un tribunal judicial.
    \item Deberá mantener información consistente acerca del estado legal de los espacios publicitarios para que un vendedor pueda ofrecerlo.
    
\end{itemize}
\item[Procesos en los que participa:]\cdtEmpty
\begin{itemize}
    \item Compra y renovación de seguros de protección para espectaculares para espectaculares.
    \item Registro de los oficios expedidos por parte del gobierno donde se autoriza el uso de los espacios para fines publicitarios, así controlar el estado de los espectaculares.
\end{itemize}
\end{Usuario}

\begin{Usuario}{\subsection{Gerente de Ventas}}{
La responsabilidad del gerente de ventas será coordinar a los vendedores para así alcanzar el logro de los objetivos de la empresa, así mismo sera el encargado de la toma de decisiones estratégicas y de la  administración.
}
   \item[Responsabilidades:]\cdtEmpty
\begin{itemize}
    \item Llevar un control de cuantos y quiénes son sus agentes de ventas.
     \item Monitorear cada uno de los departamentos 
     \item Recibir las solicitudes y ordenarlas cuando no haya vendedores disponibles para atenderlas. 
     \end{itemize}
 \item[Procesos en los que participa:]\cdtEmpty
\begin{itemize}
    \item Administración de las actividades de la empresa.
    \item Coordinación de los agentes de ventas.
\end{itemize}
\end{Usuario}