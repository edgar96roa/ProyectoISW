%=========================================================
\chapter{Modelo del Alcance}
\label{cap:reqUsr}

	En este capítulo se modela el alcance del sistema. Se presentan inicialmente los Actores involucrados y sus requerimientos, especificando cuales se alcanzaron en la primera iteración y cuales serán trabajados en la segunda iteración. Después se presentan los requerimientos funcionales de esta iteración y al final se presenta el modelo Físico y Lógico del sistema.


%---------------------------------------------------------
\section{Análisis del problema}

\subsection{Problema general}
La empresa Espectaculares S.A de C.V ha estado creciendo los últimos años y actualmente cuenta con el 95\% de los espectaculares en la Ciudad de México, los clientes sólo pueden contactar a la empresa de lunes a viernes, lo cuál se convierte en una pérdida de capital para la empresa al no existir una comunicación durante esos lapsos de tiempo, aunado a esto, al tener tantos cambios en tan poco tiempo no han implementado una manera eficiente para un mejor control en su información.%RC

\subsection{Problemas específicos}
El manejo actual de la información se maneja por separado es por ello que:
\begin{itemize}
    \item No se cuenta con la información de cuantos espectaculares se tienen.
    \item No existe un control total de la información de los espectaculares asegurados.
    \item Se desconoce que anuncios hay que retirar cuando se vence un contrato.
    \item Se desconoce con cuantos clientes cuenta la empresa.
    \item Se desconoce si los clientes han cambiado de número, dirección o nombre de la persona responsable de los contratos.
    \item El personal de infraestructura desconoce las condiciones de los espectaculares
    \item El personal de instalación y mantenimiento no tiene conocimiento de las condiciones en las cuales se encuentra un espectacular que requiera algún servicio.
    \item El personal jurídico desconoce cuales son los próximos permisos a renovar.
    \item El agente de ventas no cuenta con la información de los espectaculares disponibles.
\end{itemize}

\subsection{Causas}
Las principales causas se deben a que la empresa:
\begin{itemize}
    \item Tiene la información dividida en distintos departamentos.
    \item Ha crecido bastante en los últimos años.
    \item Ha tenido un aumento de clientes.
    \item No tiene una forma de contacto en horas no laborales para sus empleados.
\end{itemize}

\subsection{Consecuencias}
Como consecuencias la empresa:
\begin{itemize}
    \item No ha podido crecer más de lo que ha crecido.
    \item Se ha generado una posible pérdida de clientes durante los horarios no laborables en la empresa.
    \item La comunicación entre personal de ventas y clientes únicamente se realiza durante las horas laborables para los empleados.
\end{itemize}


\subsection{Requerimientos de usuario}
\begin{table}[htbp!]
	\begin{requerimientosU}
		\FRitem{RUC-1}{Control de quejas}{Necesita un apartado donde pueda plasmar y enviar sus quejas.}{1}{\TODO}
		\FRitem{RUC-2}{Notificaciones de duración de contratos}{Necesita que se le informe la fecha de vencimiento de la renta de su espectacular, así como que se le den avisos cuando ya se vaya acercando esta fecha.}{1}{\TODO}
		\FRitem{RUC-3}{Control de dudas y sugerencias}{Necesita un apartado donde pueda enviarle sus dudas, opiniones o sugerencias a la empresa.}{1}{\TODO}
		\FRitem{RUC-4}{Detalle de contrato}{Necesita poder consultar su(s) espectacular(es) contratado(s) y el estado en que se encuentran, así como a fecha límite que para renovar su contrato (en caso de que lo desee) y la fecha de terminación de este.}{1}{\TODO}
		\FRitem{RUC-5}{Inicio de sesion}{Necesita poder tener una sesión para acceder a los datos del sistema.}{1}{\TODO}
		\FRitem{RUV-1}{Registro de clientes}{Necesita registrar nuevos clientes.}{1}{\TODO}
		\FRitem{RUV-2}{Disponibilidad de espectaculares}{Saber cuáles espectaculares están disponibles para incluirlos en un contrato.}{1}{\TODO}
		\FRitem{RUV-3}{Registro rápido de clientes existentes}{Cargar la información de sus clientes desde un archivo de Excel con un formato definido. (Delimitar el formato) }{1}{\TODO}
		\FRitem{RUV-4}{Herramienta de cotización}{Necesita una herramienta auxiliar para la cotización de un contrato.}{1}{\TODO}
		\FRitem{RUV-5}{Clasificación de clientes}{Necesita conocer la clasificación en la cuál se encuentran sus clientes para otorgarles descuentos.}{1}{\TODO}
		\FRitem{RUV-6}{Control de información}{Necesita poder actualizar los datos personales de los clientes fácilmente.}{1}{\TODO}
		\FRitem{RUV-7}{Modificar contratos}{Necesita poder consultar,modificar y cancelar contratoscuando le sea necesario }{1}{\TODO}
		\FRitem{RUPI-1}{Manejo de información de espectaculares}{Necesita poder agregar, actualizar o eliminar la información relacionada a los espectaculares (ubicación, tipo, dimensiones, datos del arrendador).}{1}{\TODO}
		\FRitem{RUPI-2}{Incidentes en espectaculares}{Realizar la consulta de incidencias relacionadas a los espectaculares.}{1}{\TODO}
		\FRitem{RUPI-3}{Asignación de reparaciones}{Necesita una forma ágil de asignación de incidencias de espectaculares al personal de instalación y mantenimiento.}{1}{\TODO}
		%\FRitem{RUPI-4}{}{}{1}{\TODO}
		\FRitem{RUIM-1}{Instalación de espectaculares}{Saber cuáles serán los espectaculares que requieren ser instalados con su información correspondiente.}{1}{\TODO}
		\FRitem{RUIM-2}{Desinstalación de espectaculares}{Saber cuáles serán los espectaculares que requieren ser desinstalados con su información correspondiente.}{1}{\TODO}
		\FRitem{RUIM-3}{Mantenimiento de espectaculares}{Saber cuáles serán los espectaculares que requieren mantenimiento con su información correspondiente.}{1}{\TODO}
		\FRitem{RUIM-4}{Disponibilidad de espectaculares}{Mostrara la disponibiliad de un espectacular en una zona, el tipo de espectacular, sus dimensiones}{1}{\TODO}
		\FRitem{RUA-1}{Reporte de incidentes}{Reportar desperfectos de los espectaculares}{1}{\TODO}
		\FRitem{RUCH-1}{Registro de empleados}{Altas, bajar y cambios de empleados de la empresa}{1}{\TODO}
		\FRitem{RUPJ-1}{Registro de permisos}{Podrá registrar los oficios en donde recibe permiso por parte del gobierno para utilizar el espacio para fines publicitarios.}{1}{\TODO}
		\FRitem{RUPJ-2}{Visualización de permisos}{Necesita poder visualizar si los espectaculares cuentan con permiso por parte del gobierno}{1}{\TODO}
		\FRitem{RUPJ-3}{Historial de permisos}{Necesita consultar el historial de permisos por año y mes.}{1}{\TODO}
		\FRitem{RUPJ-4}{Noificación de vencimiento de permisos}{Necesita recibir avisos de espectaculares con permisos vencidos o próximos a vencer}{1}{\TODO}
		\FRitem{RUGV-1}{Cartera de clientes}{Necesita poder consultar el historial de los clientes que ha tenido la empresa}{1}{\TODO}
		
		
	\end{requerimientosU}

    \caption{Requerimientos de Usuario.}
    {\footnotesize\em Para leer correctamente esta tabla vea la leyenda en la Tabla~\ref{tbl:leyendaRF} en la página~\pageref{tbl:leyendaRF}.}
    \label{tbl:reqFunc}
\end{table}

\newpage
\section{Propuesta}
Con el fin de resolver los problemas anteriormente mencionados que enfrenta Espectaculares S.A de C.V nuestra empresa X-Force propone la creación de un software que permita mejorar y optimizar su funcionamiento haciendo que las distintas actividades que realizan los empleados día con día como son : la atención a clientes,el control de renta de espectaculares y actividades referentes a su mantenimiento,disponibilidad, instalación  entre otras cosas puedan ser llevadas a cabo mas fácilmente, de forma organizada y eficiente.


\subsection{Objetivos generales}

Desarrollar un software para la empresa Espectaculares S.A de C.V que le permita tener un mayor control, organización e información relacionada con los espectaculares y clientes, así como mejorar la comunicación con los clientes y personal de ventas.

\subsection{Objetivos específicos}%%Roa

\begin{itemize}
    \item Conocer el número de espectaculares con los cuales cuenta la empresa y sus características.
    \item Saber que espectaculares requieren mantenimiento preventivo o correctivo.
    \item Saber que espectaculares requieren una solicitud de permisos gubernamentales o cuales requieren una renovación de ellos.
    \item Conocer cuando las fechas próximas de vencimiento de contratos.
    \item Contar con un historial de contratos.
    \item Dar facilidad a los clientes de contactar a la empresa durante horas no laborables.
    \item Recibir las quejas de los clientes para darles seguimiento.
    \item Hacer que los clientes puedan conocer la ubicación y características de los espectaculares que han contratado dentro del sistema.
    \item Hacer que los clientes puedan conocer el impacto de los espectaculares de acuerdo a la zona en que deseen contratarlo.
\end{itemize}

\subsection{Descripción de la solución}
Dentro del sistema propuesto el cliente, personales de: ventas, infraestructura, capital humano, instalación y mantenimiento, así como los gerentes de ventas e infraestructura, además de el departamento jurídico y arrendadores tendrán acceso a una plataforma web por medio de la cuál podrán realizar cuando menos una de sus necesidades.\\

Para evitar la posible pérdida de clientes en las horas no laborables el sistema tendrá una manera de captar las solicitudes de los clientes para contactar a un empleado dentro del personal de ventas, asimismo los clientes podrán enviar sus quejas o sugerencias para darles seguimiento.\\

El gerente de ventas recibirá las solicitudes de los clientes y las asignará al personal de ventas.\\

El personal de ventas podrá dar de alta, hacer cambios o eliminar del sistema a los clientes con los cuáles cuenta, además contará con un asistente que le permita realizar una cotizaciones dónde pueda ver en un mapa cuáles son los espectaculares asociados a esa cotización.\\

El gerente de infraestructura podrá dar de alta, hacer cambios o eliminar del sistema a los espectaculares que tenga la empresa, de igual manera recibirá un reporte mensual de incidentes ocurridos.\\


\section{Alcance}

\subsection{Requerimientos funcionales}

\begin{enumerate}[leftmargin=2.5cm ,label={\bfseries RF-\arabic*}]
  
    % Permisos
   \item \textbf{ Inicio de sesión:} El sistema debe de contar con un inicio de sesión para el personal de instalación y mantenimiento, vendedores, clientes y arrendadores. Sólo los usuarios autorizados de esta forma podrán acceder a los datos del sistema.
   
      \textbf{Ref:} RUC-5
    
    \textbf{Prioridad:}MA
\item \textbf{ Registro de clientes:} Debe ser posible registrar los datos de los clientes (nombre,telefono,RFC, domicilio fiscal).

   \textbf{Ref:} RUV-1
    
    \textbf{Prioridad:} MA
\item \textbf{ Vencimiento de rentas:} El sistema deberá generar y mostrar  una notificación cuando se acerque la fecha de vencimiento de la renta del espectacular, al igual que cuando ya se haya vencido.

   \textbf{Ref:}RUC-1
    
    \textbf{Prioridad:}A
\item \textbf{ Consulta de anuncios contratados:} El sistema debe de mostrar al cliente en un mapa los espectaculares contratados a la empresa.

   \textbf{Ref:}RUC-4
    
    \textbf{Prioridad:} M
    
\item \textbf{ Envío de correos:} El sistema permitirá el envió de correos electrónicos que contengan el reporte  del estado del espectacular.

 \textbf{Ref:}RUA-1
    
    \textbf{Prioridad:} A
\item \textbf{ Historial de clientes:} El sistema debe permitir consultar el historial de clientes que se han tenido, así como la posibilidad de filtrar que clientes se tuvieron en determinadas fechas.

 \textbf{Ref:}RUGV-1
 
\textbf{Prioridad:} M

\item \textbf{ Respaldo:} La información almacenada en el sistema deberá respaldarse cada 24 horas.
\item \textbf{CRUD Contratos:} El sistema deberá permitir la creación, modificación, consulta y eliminación de contratos. 

 \textbf{Ref:}RUV-7
 
\textbf{Prioridad:} A
\item \textbf{ Información de cliente:} El sistema deberá proporcionar de manera consistente la información del cliente .
\item \textbf{ Notificación de desperfectos en espectaculares:}El sistema deberá de ser capaz de recibir un informe de desperfecto en un espectacular y notificar al personal de instalación y mantenimiento para su reparación.

 \textbf{Ref:}RUIM-3
 
\textbf{Prioridad:} A
\item \textbf{ Agregar cuenta:} El sistema deberá permitir al personal correspondiente agregar una cuenta de usuario para hacer uso del sistema.
\item \textbf{Recuperar cuenta:} El sistema deberá ofrecer la opción de recuperar password para iniciar sesión de nuevo.
\item \textbf{ Actualización de datos:} El sistema permitirá la actualización de los datos de los usuarios que interactuén con el sistema.

\item \textbf{CRUD Clientes} El sistema permitira la creación, modificación y eliminación de clientes.

 \textbf{Ref:}RUV-6
 
\textbf{Prioridad:} A
\item \textbf{CRUD Ventas} El sistema permitira la creacion, modificacion y eliminacion del personal de ventas.
\item \textbf{CRUD Instalación y Mantenimiento:} El sistema permitirá la creación, modificación y eliminación del personal de instalación y mantenimiento.
\item \textbf{ Geolocalizacion:} El sistema permitirá mostrar la información de un espectacular (Localización, información, estado).
\item \textbf{CRUD Espectaculares:} El sistema permitirá la agregación, modificación y eliminación de espectaculares o su información para actualizar el mapa de espectaculares dentro de la empresa.
%%%%%
\item \textbf{ Estado de un espectacular:} El sistema permitirá visualizar el estaDo de un espectacular en el que se encuentre ocupado, en mantenimiento y cuando ya este disponible

\end{enumerate}
\subsection{Requerimientos no funcionales}
\begin{itemize}
\item La aplicación debe ser compatible con todas las versiones de Windows, desde Windows 7.
\item Visualización en los navegadores más comunes como son Edge,Firefox,Chrome y Opera.
\item Responsive, es decir, que sea visualizable en todos los dispositivos.
\item Se utilizará el gestor de base de datos MySql.
\item Las consultas que realice el sistema deberán responder de forma inmediata.
\item La aplicación debe ser amigable con el usuario al momento de la interacción entre usuario y sistema.
\end{itemize}
\subsubsection{Plataforma}
\subsubsection{Interacción con el usuario}
\subsubsection{Procesos y reglas de negocio}
\subsubsection{Información y datos}
\subsubsection{Propiedades del software}

%============================================================================
%============================================================================
%============================================================================
%
% Tenemos que pasar todo a la estructura de arriba
%
%============================================================================
%============================================================================
%============================================================================

















\cdtInstrucciones{
	Identifique los actores que estarán involucrados en los procesos relacionados con el sistema para esta iteración de desarrollo. Ponga énfasis en los procesos involucrados.
}

%\subsection{Organigrama de la Empresa}
%\begin{figure}
%    \centering
%    \includegraphics[width=.8\textwidth]{images/organigramaEm}
%    \caption{Organigrama de la Mueblería Qetzal S. A. de C. V.}
%    \label{fig:organigrama}
%\end{figure}

%---------------------------------------------------------
\newpage


%---------------------------------------------------------


%\cdtInstrucciones{
	%Identifique y describa los requerimientos funcionales del sistema señalando: id, nombre, descripción y %prioridad.
%}


%---------------------------------------------------------
\section{Especificación de plataforma}	

\cdtInstrucciones{
	Coloque un diagrama y su descripción para aclarar el tipo de solución propuesta. \\
	
 En esta sección se debe aclarar:
	
\begin{description}
	\item[Tipo de sistema:] Web, aplicación móvil, de escritorio, híbrida, etc.
	\item[Software requerido:] Programas que se deberán instalar, desde el sistema operativo, compiladores, interpretes, servidores, etc.
	\item[Hardware requerido:] CPU, núcleos, velocidad, memoria, disco duro, etc.
	\item[servicios:] De conexión, seguridad, firewall, respaldo de energía, redundancia, uso de raids, etc.
\end{description}
}

\begin{figure}[htbp!]
	\begin{center}
		\fbox{\includegraphics[width=.6\textwidth]{images/arquitectura}}
		\caption{Arquitectura del sistema.}
		\label{fig:arquitectura}
	\end{center}
\end{figure}

En la figura~\ref{fig:arquitectura} se describe la estructura del sistema, en ella se detalla ...


