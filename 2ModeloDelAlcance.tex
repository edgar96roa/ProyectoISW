%=========================================================
\chapter{Modelo del Alcance}
\label{cap:reqUsr}

	En este capítulo se modela el alcance del sistema. Se presentan inicialmente los Actores involucrados y sus requerimientos, especificando cuales se alcanzaron en la primera iteración y cuales serán trabajados en la segunda iteración. Después se presentan los requerimientos funcionales de esta iteración y al final se presenta el modelo Físico y Lógico del sistema.


%---------------------------------------------------------
\section{Modelado de Usuarios}
\cdtInstrucciones{
	Identifique los actores que estarán involucrados en los procesos relacionados con el sistema para esta iteración de desarrollo. Ponga énfasis en los procesos involucrados.
}

%\subsection{Organigrama de la Empresa}
%\begin{figure}
%    \centering
%    \includegraphics[width=.8\textwidth]{images/organigramaEm}
%    \caption{Organigrama de la Mueblería Qetzal S. A. de C. V.}
%    \label{fig:organigrama}
%\end{figure}

%---------------------------------------------------------
\newpage
\begin{Usuario}{\subsection{Cliente}}{
   Definimos al cliente como la persona o empresa externa a la compañía la cual acude a la empresa en búsqueda de una solución de marketing para sus necesidades.\\
   
}
    \item[Responsabilidades:]\cdtEmpty 
   \begin{itemize}
       \item Solicitar un plan de marketing para la empresa a la cuál está asociado.
       \item Identificar la cantidad de personas a las cuales el espectacular deberá impactar.
       \item Encargado de definir el presupuesto con el que cuenta para que el vendedor pueda brindarle un plan a su medida.
       \item Respetar al presupuesto acordado con el vendedor.
	   \item Respetar las fechas de inicio y finalización de contrato.
   \end{itemize}
	\item[Procesos en los que participa:] \cdtEmpty
    \begin{itemize}
		\item Contratación, renovación y cancelación de contratos de servicios de marketing.
	    \item Monitoreo de historial y estatus de servicios contratados.
    \end{itemize}
\end{Usuario}

\begin{Usuario}{\subsection{Vendedor}}{
	El vendedor es el encargado de mostrar al cliente las diferentes opciones de publicidad que la empresa ofrece, con el fin de hacer que el cliente encuentre alguna que se ajuste a sus necesidades y presupuestos.\\
}
    \item[Responsabilidades:] \cdtEmpty
    \begin{itemize}
		\item Mostrar al cliente las opciones de publicidad con las que la empresa cuenta.
        \item Encargado de la negociación con el cliente.
        \item Cerrar un trato que beneficie al cliente y genere ganancias a la empresa.
        \item Interacción con los posibles clientes para la realización de un contrato.
    \end{itemize}

	\item[Procesos en los que participa:] \cdtEmpty
    \begin{itemize}
		\item Altas, bajas y cambios de información de clientes.
		\item Cotización de contratos.
    \end{itemize}
\end{Usuario}

\begin{Usuario}{\subsection{Infraestructura}}{
	El personal de infraestructura es el encargado de realizar nuevas adquisiciones de espectaculares para la empresa además de actualizar su información pertinente.\\
}
    \item[Responsabilidades:] \cdtEmpty
    \begin{itemize}
        \item Gestionar la información congruente a los espectaculares.
		\item Adquisición de espacios disponibles para la instalación de nuevos espectaculares.
		\item Asignación de tareas al personal de mantenimiento e instalación.
		\item Monitoreo del estatus de los seguros de los espectaculares.
		\item Comunicación continua con las personas arrendadoras de espacios para espectaculares.
	\end{itemize}

	\item[Procesos en los que participa:] \cdtEmpty
    \begin{itemize}
		\item Adquisición de espectaculares.
		\item Elaboración de reporte de incidencias.
		\item Altas, bajas y cambios de información de espectaculares.
    \end{itemize}
\end{Usuario}

%\begin{Usuario}{\subsection{Gerente de Ventas}}{
%	Es el encargado de todas las operaciones de ventas al mayoreo y al menudeo. coordina y supervisa el trabajo %de los Agentes de Ventas y Encargados de Tienda.
%	Reporta directamente al Gerente de Operaciones
%}
 %   \item[Responsabilidades:] \cdtEmpty
 %   \begin{itemize}
%		\item Supervisar la operación de ventas.
%		\item Plantear y supervisar el logro de las metas de ventas de la empresa y su crecimiento económico.
%		\item ...
%    \end{itemize}

%	\item[Perfil:] \cdtEmpty
%    \begin{itemize}
%		\item Amplia experiencia en el ramo.
%		\item Licenciatura como mínimo.
%		\item ...
%    \end{itemize}
%	\item[Procesos en los que participa:] \cdtEmpty
 %   \begin{itemize}
%		\item PC-V01 Aprobar las ordenes de compra al mayoreo.
%		\item PC-V02 Supervisar las ventas al menudeo.
%		\item PC-V03 Elaborar informe de ventas mensual.
%		\item ...
%    \end{itemize}
%\end{Usuario}

%---------------------------------------------------------
\begin{Usuario}{\subsection{Instalación y Mantenimiento}}{
    Es el personal encargado de la mano de obra referente a los espectaculares, es decir instalación, desinstalación, atención a incidentes.\\
}
    \item[Responsabilidades:] \cdtEmpty
    \begin{itemize}
        \item Realización de instalación, desinstalación y mantenimiento a espectaculares.
		\item Notificar los servicios que realiza.
		\item Atención pronta a reportes de incidencias.
    \end{itemize}

	\item[Procesos en los que participa:] \cdtEmpty
    \begin{itemize}
		\item Instalación, desinstalación y mantenimiento a espectaculares.
		\item Actualización del estatus de un espectacular.
    \end{itemize}
\end{Usuario}

\begin{Usuario}{\subsection{Arrendador}}{
    Persona encargada de negociar el uso de un espacio de su propiedad para uso de un espectacular.\\
}
    \item[Responsabilidades:] \cdtEmpty
    \begin{itemize}
        \item Revisón constante del estado de los espectaculares que tenga a su nombre en un contrato de arrendamiento.
        \item Reportar algún incidente relacionado con el espectacular.
        \item Respetar los plazos establecidos en la renta del espacio para el espectacular.
    \end{itemize}
    
    \item[Procesos en los que participa:] \cdtEmpty
    \begin{itemize}
        \item Reporte de desperfectos en espectaculares.
        %\item Validación del estatus de un espectacular.
    \end{itemize}
\end{Usuario}

\begin{Usuario}{\subsection{Capital humano}}{
La responsabilidad de capital humano sera la gestión de información de empleados que requieren tener acceso al sistema.
}
\item[Responsabilidades:]\cdtEmpty
\begin{itemize}
    \item Manejo de información congruente al personal de la empresa.
\end{itemize}
\item[Procesos en los que participa:]\cdtEmpty
\begin{itemize}
    \item Responsable de creación de usuarios al personal dentro de la empresa que necesite acceder al sistema.
    \item Responsable de cambios y eliminación de información correspondiente de los empleados dentro del sistema.
\end{itemize}
\end{Usuario}

\begin{Usuario}{\subsection{Área Jurídica}}{
La actividad que desempeña el área jurídica esta relacionada con los permisos y seguros relacionados a los espectaculares.
}
\item[Responsabilidades:]\cdtEmpty
\begin{itemize}
    \item Gestión de los permisos y seguros de espectaculares.
    \item Garantizar el cumplimiento de las regulaciones y las leyes.
    \item Representar a la empresa en asuntos legales de diversa índole ante un tribunal judicial.
\end{itemize}
\item[Procesos en los que participa:]\cdtEmpty
\begin{itemize}
    \item Compra y renovación de seguros de protección para espectaculares para espectaculares.
\end{itemize}
    
\end{Usuario}
\newpage

%---------------------------------------------------------
\section{Requerimientos de usuario}

%\cdtInstrucciones{
	%Identifique y describa los requerimientos funcionales del sistema señalando: id, nombre, descripción y %prioridad.
%}

\begin{table}[htbp!]
	\begin{requerimientosU}
		\FRitem{RUC-1}{Control de información}{Necesita poder actualizar sus datos personales fácilmente.}{1}{\TODO}
		\FRitem{RUC-2}{Notificaciones de duración de contratos}{Necesita que se le informe la fecha de vencimiento de la renta de su espectacular, así como que se le den avisos cuando ya se vaya acercando esta fecha.}{1}{\TODO}
		\FRitem{RUC-3}{Control de dudas y sugerencias}{Necesita un apartado donde pueda enviarle sus dudas, opiniones o sugerencias a la empresa.}{1}{\TODO}
		\FRitem{RUC-4}{Detalle de contrato}{Necesita poder consultar su(s) espectacular(es) contratado(s) y el estado en que se encuentran, así como a fecha límite que para renovar su contrato (en caso de que lo desee) y la fecha de terminación de este.}{1}{\TODO}
		\FRitem{RUV-1}{Registro de clientes}{Necesita registrar nuevos clientes.}{1}{\TODO}
		\FRitem{RUV-2}{Disponibilidad de espectaculares}{Saber cuáles espectaculares están disponibles para incluirlos en un contrato.}{1}{\TODO}
		\FRitem{RUV-3}{Registro rápido de clientes existentes}{Cargar la información de sus clientes desde un archivo de Excel con un formato definido. (Delimitar el formato) }{1}{\TODO}
		\FRitem{RUV-4}{Herramienta de cotización}{Necesita una herramienta auxiliar para la cotización de un contrato.}{1}{\TODO}
		\FRitem{RUV-5}{Clasificación de clientes}{Necesita conocer la clasificación en la cuál se encuentran sus clientes para otorgarles descuentos.}{1}{\TODO}
		\FRitem{RUPI-1}{Manejo de información de espectaculares}{Necesita poder agregar, actualizar o eliminar la información relacionada a los espectaculares (ubicación, tipo, dimensiones, datos del arrendador).}{1}{\TODO}
		\FRitem{RUPI-2}{Incidentes en espectaculares}{Realizar la consulta de incidencias relacionadas a los espectaculares.}{1}{\TODO}
		\FRitem{RUPI-3}{Asignación de reparaciones}{Necesita una forma ágil de asignación de incidencias de espectaculares al personal de instalación y mantenimiento.}{1}{\TODO}
		%\FRitem{RUPI-4}{}{}{1}{\TODO}
		\FRitem{RUIM-1}{Instalación de espectaculares}{Saber cuáles serán los espectaculares que requieren ser instalados con su información correspondiente.}{1}{\TODO}
		\FRitem{RUIM-2}{Desinstalación de espectaculares}{Saber cuáles serán los espectaculares que requieren ser desinstalados con su información correspondiente.}{1}{\TODO}
		\FRitem{RUIM-3}{Mantenimiento de espectaculares}{Saber cuáles serán los espectaculares que requieren mantenimiento con su información correspondiente.}{1}{\TODO}
		\FRitem{RUA-1}{Reporte de incidentes}{Reportar desperfectos de los espectaculares}{1}{\TODO}
		\FRitem{RUCH-1}{Registro de empleados}{Altas, bajar y cambios de empleados de la empresa}{1}{\TODO}
		\FRitem{RUPJ-1}{Registro de permisos}{Podrá registrar los oficios en donde recibe permiso por parte del gobierno para utilizar el espacio para fines publicitarios.}{1}{\TODO}
		\FRitem{RUPJ-2}{Visualización de permisos}{Necesita poder visualizar si los espectaculares cuentan con permiso por parte del gobierno}{1}{\TODO}
		\FRitem{RUPJ-3}{Historial de permisos}{Necesita consultar el historial de permisos por año y mes.}{1}{\TODO}
		\FRitem{RUPJ-4}{Noificación de vencimiento de permisos}{Necesita recibir avisos de espectaculares con permisos vencidos o próximos a vencer}{1}{\TODO}	
	\end{requerimientosU}
    \caption{Requerimientos funcionales del sistema.}
    {\footnotesize\em Para leer correctamente esta tabla vea la leyenda en la Tabla~\ref{tbl:leyendaRF} en la página~\pageref{tbl:leyendaRF}.}
    \label{tbl:reqFunc}
\end{table}

%---------------------------------------------------------
\section{Especificación de plataforma}	

\cdtInstrucciones{
	Coloque un diagrama y su descripción para aclarar el tipo de solución propuesta. \\
	
 En esta sección se debe aclarar:
	
\begin{description}
	\item[Tipo de sistema:] Web, aplicación móvil, de escritorio, híbrida, etc.
	\item[Software requerido:] Programas que se deberán instalar, desde el sistema operativo, compiladores, interpretes, servidores, etc.
	\item[Hardware requerido:] CPU, núcleos, velocidad, memoria, disco duro, etc.
	\item[servicios:] De conexión, seguridad, firewall, respaldo de energía, redundancia, uso de raids, etc.
\end{description}
}

\begin{figure}[htbp!]
	\begin{center}
		\fbox{\includegraphics[width=.6\textwidth]{images/arquitectura}}
		\caption{Arquitectura del sistema.}
		\label{fig:arquitectura}
	\end{center}
\end{figure}

En la figura~\ref{fig:arquitectura} se describe la estructura del sistema, en ella se detalla ...


