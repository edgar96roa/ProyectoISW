                                                                                                                                                                                                                                                                                                                                                                                                                                                                                                                                                                                                                                                                                                                                                                                                                                                                                                                                                                                                                                                                                                                                                                                                                                                                                                                                                                                                                                                                                                                                                                                                                                                                                                                                                                                                                                       %=========================================================
\chapter{Introducción}
%---------------------------------------------------------
\section{Presentación}
Este proyecto surge de la necesidad que presenta una empresa dedicada a la instalación y renta de anuncios espectaculares, ha crecido en los últimos años y requieren de un mejor control en sus actividades como son el número de espectaculares que se encuentran en funcionamiento, los que necesitan mantenimiento, o los cuales ya superaron su fecha de vencimiento de contrato.
Es por ello que nuestro equipo ha pensado en un software  que pueda satisfacer algunas necesidades que la empresa presenta y por lo tanto ayudar al personal de ventas a ofrecer el mejor servicio a los clientes.


%\cdtInstrucciones{
%	Indique el propósito del documento y las distintas formas en que puede ser utilizado.\\
%}
%	Este documento contiene la especificación de los requerimientos del usuario y del sistema del sistema a %desarrollar. Tiene como objetivo establecer la naturaleza y funciones del sistema para su evaluación al final %del semestre. Este documento debe ser aprobado por los principales responsables del proyecto.
%	
%	Este documento es el C2-EP1 del proyecto ``{\em Nombre del proyecto}''.
	
%---------------------------------------------------------
\section{Organización del contenido}

	En el capítulo \ref{cap:reqUsr} ...
	
	En el capítulo \ref{cap:reqSist} ...

%---------------------------------------------------------
\section{Notación, símbolos y convenciones utilizadas}

Los requerimientos del usuario utilizan una clave RUN-X, donde:
	
\begin{description}
    \item[RU] Es la clave para los requerimientos de usuario.
	\item[X] Toma los valores \textbf{C}, \textbf{PIM}, \textbf{V}, \textbf{CH}, \textbf{GV} para los actores
	\begin{description}
	    \item[C] Cliente
	    \item[PIM] Personal de instalación y mantenimiento
	    \item[V] Vendedores
	    \item[CH] Capital humano
	    \item[A] Arrendador
	     \item[GV] Gerente de Ventas
	\end{description}
	\item[N] Es un número consecutivo: 1, 2, 3, ...
\end{description}
Los requerimientos funcionales utilizan una clave \textbf{RF-X}, donde:
\begin{description}
    \item[X] Es un número consecutivo: 1, 2, 3, ...
    \item[RF] Es la clave para todos los Requerimientos Funcionales.
\end{description}
	Además, para los requerimientos funcionales se usan las abreviaciones que se muestran en la tabla~\ref{tbl:leyendaRF}.
\begin{table}[hbtp!]
	\begin{center}
    \begin{tabular}{|r l|}
	    \hline
    	{\footnotesize Id} & {\footnotesize\em Identificador del requerimiento.}\\
    	{\footnotesize Pri.} & {\footnotesize\em Prioridad}\\
    	{\footnotesize Ref.} & {\footnotesize\em Referencia a los Requerimientos de usuario.}\\
    	{\footnotesize MA} & {\footnotesize\em Prioridad Muy Alta.}\\
    	{\footnotesize A} & {\footnotesize\em Prioridad Alta.}\\
    	{\footnotesize M} & {\footnotesize\em Prioridad Media.}\\
    	{\footnotesize B} & {\footnotesize\em Prioridad Baja.}\\
    	{\footnotesize MB} & {\footnotesize\em Prioridad Muy Baja.}\\
		\hline
    \end{tabular} 
    \caption{Leyenda para los requerimientos funcionales.}
    \label{tbl:leyendaRF}
	\end{center}
\end{table}